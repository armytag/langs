\documentclass[a4paper,12pt,twoside,openright]{memoir}

%------------------------------------------------------
%   PACKAGES SET-UP
%------------------------------------------------------

%---------------
%   PAGE
%---------------

\usepackage{geometry}   % Required for adjusting page dimensions and margins

\geometry{
	paper=a4paper,      % Using standard european paper size, 'letterpaper' for US letter size
	top=18mm,           % Top margin (it was 18mm before, in case I fucked this up)
	bottom=12.5mm,        % Bottom margin (it was 13mm before, in case I fucked this up) % canonically changed to 12.5
%	left=21mm,          % Left margin
%	right=21mm,         % Right margin
    inner=19mm,         % inner margin
    outer=21mm,         % outer margin
	headheight=19mm,    % Header height
	%includehead,       % -- Without these, the header and footer are --
	%includefoot,       % -- not included within the area defined by margins --
	%footskip=14mm,     % Space from the bottom margin to the baseline of the footer
	%headsep=10mm,      % Space from the top margin to the baseline of the header
	%showframe,         % Uncomment to show how the type block is set on the page
}

\usepackage{xcolor}     % Required for specifying colors by name
\definecolor{mainoffwhite}{RGB}     {244, 239, 230}     % Main colour for pages and some text
\definecolor{corneryellow}{RGB}     {250, 183, 84 }     % Corners, some important things
\definecolor{purplevivid}{RGB}      {162, 92 , 156}     % Inserts
\definecolor{purplight}{RGB}        {180, 149, 184}     % Inserts highlights
\definecolor{somegrey}{RGB}         {106, 109, 114}     % Some stuff's backdrop
\definecolor{almostblack}{RGB}      {38 , 37 , 44 }     % main text colour
\definecolor{prettygreen}{RGB}      {101, 194, 149}     % A pretty green for highlighting text
\definecolor{chalgreen}{RGB}        {142, 211, 124}     % Background color for Challenge page
\definecolor{watermelony}{RGB}      {234, 80 , 151}     % Watermelony Challenge color

\usepackage[pagecolor={mainoffwhite}, nopagecolor={mainoffwhite}]{pagecolor} % provides \thepagecolor, better usage for colour consistency

\usepackage{fancyhdr}   % Fancy headers and footers!
\pagestyle{fancy}       % Set page style to use fancyhdr
\fancyhf{}

\renewcommand\chaptermark[1]{\markboth{\headingfont #1}{}}

\newsavebox{\myheadbox }% Heading storage box for centering headers in yellowcorner


\renewcommand\headrulewidth{0pt} % no line between document and header
\fancyhead{} % clear header
%\fancyhead[RE]{\nouppercase\headingfont\bfseries\fontsize{16pt}{16pt} \leftmark}
\fancyhead[RE]{\savebox{\myheadbox}{\hspace{75mm}}\makebox[\wd\myheadbox][c]{\nouppercase\headingfont\bfseries\fontsize{16pt}{16pt} \leftmark}} %\usebox{\myheadbox}
\fancyfoot{} % clear footer
\fancyfoot[RE,LO]{\headingfont\bfseries\fontsize{10pt}{10pt} Page \thepage}

\usepackage{CharisSIL}

\usepackage{float}                              % manipulate float placement
\usepackage{tikz, tikz-qtree}                   % Required to draw custom shapes & trees
\usepackage{Required/tikzvowel}                 % Vowel charts w/ tikz
\usepackage{pgf}
\usetikzlibrary{positioning, arrows.meta}
\usetikzlibrary{backgrounds, matrix, chains, positioning, decorations.pathreplacing, arrows}
\usetikzlibrary{shapes, arrows, positioning, calc, chains, scopes}
\usepackage{afterpage}                          % For changing page colors (for cover designs)

% needed for Baerian Notation (https://github.com/LLBlumire/recursive-baerian-phonotactics-notation/blob/master/Recursive_Baerian_Syntax_Notation.pdf)
%\usepackage{amsmath}
\usepackage{mathtools}
\usepackage{mathabx}
\usepackage{relsize}
\usepackage[nogroupskip,nonumberlist]{glossaries}
\usepackage{trimspaces}
\usepackage{glossary-mcols}

\usepackage{multicol}

\usepackage{array}                                          % Enables the following,
\newcolumntype{$}{>{\global\let\currentrowstyle\relax}}     % which is really just some automated formatting
\newcolumntype{^}{>{\currentrowstyle}}                      % for tables to have their headers in bold
\newcommand{\rowstyle}[1]{\gdef\currentrowstyle{#1}%
  #1\ignorespaces
}

% Required for including pictures
\usepackage{graphicx}
\graphicspath{ {./Pictures/} }
\usepackage{background}             % Required to draw and plac eiamges in background

\usepackage{ifthen}                 % Conditions!

\usepackage{lingmacros}     % Various useful things some may be used to when writing for conlanging
\usepackage{enumitem}       % Better lists
\setlist{noitemsep,nosep}   % Reduce spacing between bullet points and numbered lists, and between the items

\usepackage{lipsum}         % Inserts dummy text

\usepackage{titlesec}       % Allows to style the titles of sections and all
\usepackage{titletoc}       % Allows manipulation of the ToC

% baarux package and setup from u/akamchinjir
\usepackage[mcolblock,nostandards]{Required/baabbrevs}
\usepackage[canonical,nochapterlabels]{Required/baarux-0.9.11}
\baaruset{junctureoutersep=0.4ex, junctureinnersep=0.2ex}
\baaruset{hangindent=1.5em} % adjusting indenting on multiline glosses, specifically for HidingWaters
\renewcommand{\glossarypreamble}{\thispagestyle{empty}} % for removing page number from abbreviations glossary

\newbaarulinetype{tr}{lit}{\textit{lit.} \baarudquote}
\newbaarulinetype{tr}{alttr}{Or: \baarudquote}

% sets example counter to only count within chapters, but not show the chapter number
%\counterwithout{exesi}{chapter} % Removed this, added [nochapterlabels] to package options instead

\usepackage{verbatimbox}

\usepackage{tabu} % Tables!

\usepackage[hidelinks,hypertexnames=false]{hyperref} % Links will not alter the format

\usepackage{bookmark} % Allows for pdfbookmark

\usepackage[scaled=.92]{helvet}%. Sans serif - Helvetica
\usepackage{calc, color}

\usepackage{multirow} % For annoying tables

\usepackage{ulem}   % for better, customisable underlines


%------------------------------------------------------
%   FRILL & EMBELLISHMENTS
%------------------------------------------------------

\newcommand{\separ}{
	\begin{center}
		\noindent\rule{0.7\textwidth}{0.4pt}
	\end{center}
}

\newfontfamily\playfair[Path=Required/Fonts/]{PlayfairDisplay.ttf}
\DeclareTextFontCommand{\pffont}{\playfair}

\newfontfamily\latolight[Path=Required/Fonts/]{Lato-Light.ttf}
\DeclareTextFontCommand{\llfont}{\latolight}

\newfontfamily\nototc[Path=Required/Fonts/]{NotoSerifTC-Medium.otf}
\DeclareTextFontCommand{\tcfont}{\nototc}

\newfontfamily\karma[Path=Required/Fonts/]{Karma-Medium.ttf}
\DeclareTextFontCommand{\kmfont}{\karma}

\usepackage{xeCJK} % for East Asian scripts, must be loaded with XeLaTeX

\usepackage{tabto}

\NumTabs{5}

\newcommand{\scell}[2][c]{%
  \begin{tabular}[#1]{@{}c@{}}#2\end{tabular}} % Used for line breaks within a cell; [t] aligns with top, [b] aligns with bottom

%------------------------------------------------------
%   TABLE OF CONTENTS & BAABBREVS GLOSSARY
%------------------------------------------------------
\usepackage{pdfpages} % for inserting full-page pdf?

\usepackage{tocloft}

\makeglossaries

\makeatletter
\renewcommand{\contentsname}{Showcases}
\renewcommand{\cftchapteraftersnum}{ | }
\renewcommand{\cftchapterfont}{\headingfont}
\renewcommand{\cftchapterleader}{\bfseries\cftdotfill{\cftdotsep}}
\renewcommand{\cftchapterpagefont}{\headingfont}
\makeatother

\newlength{\mylen}   % a "scratch" length
\settowidth{\mylen}{\cftchapterpresnum\cftchapteraftersnum} % extra space
\addtolength{\cftchapternumwidth}{\mylen} % add the extra space


%------------------------------------------------------
%   TITLE STYLING
%------------------------------------------------------

% \newfontfamily\headingfont[]{Arial} % creates a new font command we'll use for titles and headings
% \newfontfamily\latofont[]{Lato}     % creates a new font command we'll use for author info
\newfontfamily\headingfont[]{NotoSans-Regular} % creates a new font command we'll use for titles and headings
\newfontfamily\latofont[]{NotoSans-Regular}    % creates a new font command we'll use for author info
\newcommand{\titlesize}{\fontsize{32pt}{32pt}\selectfont}   % creates a command to set font size to 32pt
\newcommand{\authorsize}{\fontsize{16pt}{16pt}\selectfont}  % creates a command to set font size to 16pt



\makeatletter
\def\@maketitle{
  %\newpage
  \null
  %\vskip .5em
  \begin{center}
  %\let \footnote \thanks
    {\headingfont\titlesize\bfseries \@title \par}
    \vskip .5em
    {%\large
      %\lineskip .5em
      \begin{tabular}[t]{c}
        {\latofont\authorsize\bfseries by \textit{\@author}}
      \end{tabular}\par}
    \vskip 1em
    %{\large \@date}
    %\vskip 1em
    {\headingfont\authorsize\bfseries \subtitle}
  \end{center}
  \par
  \vskip 1.5em}
\makeatother

%---------------
%   TITLE
%---------------


%---------------
%   SUBTITLE
%---------------



%---------------
%   AUTHOR
%---------------


%------------------------------------------------------
%   GRAPHICS
%------------------------------------------------------



%---------------
%   HEADER
%---------------

\renewcommand{\headrulewidth}{0.0pt}

% Defining shapes with tikz
\tikzset{
    hdrmain/.style={ % requires library shapes.geometric
        draw,
        fill=corneryellow,
%        inner sep=mm,
%        outer sep=0mm,
        trapezium,
        trapezium left angle=145,
        trapezium right angle=90,
        minimum height=13mm,
        minimum width= 123mm,
%        text width=45mm,
        text centered,
    },
}
%        \begin{tikzpicture}
%            \node[hdrmain](hdr){\color{almostblack}Plenty of stuff};
%        \end{tikzpicture}

\newif\ifBgUse % Lys's test for removing bg from pages

\backgroundsetup{scale=1,angle=0,opacity=1,contents={}}
\AddEverypageHook{%
    \ifBgUse % Lys's test for removing bg from pages
    \backgroundsetup{contents={}
}%  Set an if/else conditional statement that checks whether the page is odd or not
    \ifthenelse{\isodd{\value{page}}}
        {% IF YES
            \SetBgPosition{current page.north west}
            \SetBgContents{
                \begin{tikzpicture}[remember picture,overlay,xshift=0mm,yshift=0mm]
                    \fill[color=corneryellow](0mm,-297mm) -- (0mm,-290.5mm) -- (95.8mm,-290.5mm) -- (105mm,-297mm) -- cycle;
                \end{tikzpicture}
            }
            \BgMaterial%
        }%
        {% IF NOT
            \SetBgPosition{current page.north east}
            \SetBgContents{
                \begin{tikzpicture}[remember picture,overlay,xshift=0mm,yshift=0mm]
                    \fill[color=corneryellow](0mm,0mm) -- (0mm,-13mm) -- (-105mm,-13mm) -- (-123.5mm,-0mm) -- cycle;
                    \fill[color=corneryellow](0mm,-297mm) -- (0mm,-290.5mm) -- (-52.5mm,-290.5mm) -- (-61.7mm,-297mm) -- cycle;
                \end{tikzpicture}
            }
            \BgMaterial%
        }%
        \fi
}



%---------------
%   CHALLENGE PAGE BACKGROUND STYLE
%---------------

\newif\ifBgUseTwo % Lys's test for removing bg from pages

\backgroundsetup{scale=1,angle=0,opacity=1,contents={}}
\AddEverypageHook{%
    \ifBgUseTwo % Lys's test for removing bg from pages
    \backgroundsetup{contents={}
}%  Set an if/else conditional statement that checks whether the page is odd or not
    \ifthenelse{\isodd{\value{page}}}
        {% IF YES
            \SetBgPosition{current page.north west}
            \SetBgContents{
                \begin{tikzpicture}[remember picture,overlay,xshift=0mm,yshift=0mm]
                    \fill[color=chalgreen](0mm,-297mm) -- (0mm,-290.5mm) -- (95.8mm,-290.5mm) -- (105mm,-297mm) -- cycle;
                \end{tikzpicture}
            }
            \BgMaterial%
        }%
        {% IF NOT
            \SetBgPosition{current page.north east}
            \SetBgContents{
                \begin{tikzpicture}[remember picture,overlay,xshift=0mm,yshift=0mm]
                    \fill[color=chalgreen](0mm,0mm) -- (0mm,-13mm) -- (-105mm,-13mm) -- (-123.5mm,-0mm) -- cycle;
                    \fill[color=chalgreen](0mm,-297mm) -- (0mm,-290.5mm) -- (-52.5mm,-290.5mm) -- (-61.7mm,-297mm) -- cycle;
                \end{tikzpicture}
            }
            \BgMaterial%
        }%
        \fi
}


%------------------------------------------------------
%   HEADINGS STYLING
%------------------------------------------------------

\newcommand{\sectionnamesize}{\fontsize{40pt}{40pt}\selectfont}
\newcommand{\partsize}{\fontsize{32pt}{32pt}\selectfont} % Creates a command that sets font size to X
\newcommand{\chaptersize}{\fontsize{26pt}{26pt}\selectfont} % Creates a command that sets font size to Y
\newcommand{\headingonesize}{\fontsize{18pt}{18pt}\selectfont} % Creates a command that sets font size to 18
\newcommand{\subheadingsize}{\fontsize{15pt}{15pt}\selectfont} % Creates a command that sets font size to 15
\newcommand{\subsubheadingsize}{\fontsize{13pt}{13pt}\selectfont} % Creates a command that sets font size to 15

\newcommand{\segsize}{\fontsize{64pt}{64pt}\selectfont} % command for Segments title size
\newcommand{\issuesize}{\fontsize{20pt}{20pt}\selectfont}

%---------------
%   CHAPTER
%---------------

\makeatletter
  \renewcommand{\thechapter}{\two@digits{\value{chapter}}}
\makeatother

\makechapterstyle{box}{
  \renewcommand*{\printchaptername}{}
  \renewcommand*{\chapnumfont}{\headingfont\bfseries\fontsize{40pt}{40pt}\selectfont}
  \renewcommand*{\printchapternum}{
    \flushleft
    \begin{tikzpicture}
      \draw[fill,color=almostblack, xshift=-26mm] (0mm,0mm) rectangle (10.5mm, 26mm); % big box
      \draw[color=almostblack, xshift=-12mm] (10.5mm,13mm) node { \chapnumfont\thechapter }; % chapter number
      \draw[fill, color=almostblack, xshift=13mm] (0mm, 0mm) rectangle (0.8mm,26mm);   % separator
      %\draw[color=almostblack, xshift=26mm] (26mm,13mm) node {\printchaptertitle{test}};
    \end{tikzpicture}
  }
  \renewcommand*{\afterchapternum}{}
  \renewcommand\printchaptertitle[1]{\headingfont\chaptersize\bfseries%
    \hskip6.5mm \raisebox{10.5mm}{\parbox[s]{\textwidth-55mm}{##1}}
  }
}

\chapterstyle{box}
\newcommand{\byline}[1]{% command to add a by-line
\hskip15mm \raisebox{20mm}{\parbox{\textwidth-55mm}{\headingfont\subheadingsize by \textbf{#1}}}
\newline
}
\newcommand{\articlesub}[1]{% command to make article subtitles with the purple box (since subtitles don't seem to fit nicely with chapterstyles)
\begin{tikzpicture}
  \draw[fill,color=mainoffwhite] (0mm,0mm) rectangle (1mm,1mm);
  \draw[fill,color=purplevivid] (\textwidth+21mm,0mm) rectangle ++(-10.5mm, 13mm);
  \node[anchor=east] at (\textwidth+0mm, 5.5mm) {\headingfont\bfseries\headingonesize #1}; 
\end{tikzpicture}
}
\newcommand{\chalsub}[1]{% command for box color for challenges)
\begin{tikzpicture}
  \draw[fill,color=mainoffwhite] (0mm,0mm) rectangle (1mm,1mm);
  \draw[fill,color=watermelony] (\textwidth+21mm,0mm) rectangle ++(-10.5mm, 13mm);
  \node[anchor=east] at (\textwidth+0mm, 5.5mm) {\headingfont\bfseries\headingonesize #1}; 
\end{tikzpicture}
}



%---------------
%   SECTION
%---------------

\titleformat{\section} % we want to style the section headings only
  {\headingfont\bfseries\headingonesize}{\thesection}{1em}{}
  \titlespacing{\section}{0em}{0em}{0em}

%---------------
%   SUBSECTION
%---------------

\titleformat{\subsection} % we want to style the subsection headings only
  {\headingfont\bfseries\subheadingsize}{\thesection}{1em}{}
  \titlespacing{\subsection}{6.5mm}{0em}{0em}
 
\titleformat{\subsubsection} % we want to style the subsubsection headings only
  {\headingfont\bfseries\subsubheadingsize}{\thesection}{1em}{}
  \titlespacing{\subsubsection}{6.5mm}{0em}{0em}
  
%------------------------------------------------------
%   TEXT STYLING
%------------------------------------------------------

\setlength{\parindent}{3.2mm}   % sets the indentation of the first line of a paragraph
\setlength{\parskip}{12pt}      % sets the space between the paragraph and what's above it
\color{almostblack}

\newcommand{\highlight}[1]{{\color{purplevivid}{#1}}}

%------------------------------------------------------
%   ARTICLE-SPECIFIC STUFF
%------------------------------------------------------

% reduplication
\newbaarucmd[bits]{^}{\baarujuncture{\~{}}}

% infixes
\newbaarucmd[bits]{<}{\baaruopenbracket〈}
\newbaarucmd[bits]{>}{\baaruclosebracket〉}

% smoyding

\newcommand{\dex}[2]{\textbf{#1} \textit{`#2'}}
\def\baarusmoyd#1{(\textsc{5moyd} \##1)}
\newbaarulinetype{citation}{smoyd}{\baarusmoyd}

% ipa glossing
\def\baarubrack#1{[#1]}
\newbaarulinetype{unaligned}{ipa}{\baarubrack}

%------------------------------------------------------
%   LABELS & REF -- KUDOS TO AKAM
%------------------------------------------------------

% Wrap each article in \begin{article}{PREFIX}...\end{article}. PREFIX will
% be added to each cross-referencing label defined or used within the
% article. Two hooks, \PREFIXdefs and \endPREFIXdefs, are run at the start
% and end of the environment. The first can be used to load article-specific
% definitions or configuration, the second is probably useless.

% \makeatletter
% \AtBeginDocument{%
%   \let\seg@ref@saved\ref
%   \let\seg@label@saved\label
%   \let\seg@hyperref@saved\hyperref

%   \def\ref{\@ifstar\seg@ref@starred\seg@ref@}
%   \def\seg@ref@starred#1{\seg@ref@saved*{\segprefix#1}}
%   \def\seg@ref@#1{\seg@ref@saved{\segprefix#1}}

%   \def\label#1{\seg@label@saved{\segprefix#1}}

%   \def\hyperref{\@ifnextchar[\seg@hyperref@optarg\seg@hyperref@saved}%]
%   \def\seg@hyperref@optarg[#1]{\seg@hyperref@saved[\segprefix#1]}
% }

% \newenvironment{article}[1]{%
%   \ifcsdef{segprefix:#1}{%
%     \PackageError{segments}{Duplicate article prefix `#1'}\@eha
%   }{}%
%   \global\cslet{segprefix:#1}\relax
%   \def\segprefix{#1}%
%   \csuse{#1defs}%
% }{%
%   \csuse{end\segprefix defs}%
% }%
% \makeatother

\baabbrev[\FIRST]{1}{First person}
\baabbrev[\OBJ]{o}{Object}
\baabbrev[\SECOND]{2}{Second person}
\baabbrev[\SUBJ]{s}{Subject}
\baabbrev[\THIRD]{3}{Third person}
\baabbrev{1r}{Relating to first person}
\baabbrev{abl}{Ablative}
\baabbrev{abst}{Abstraction}
\baabbrev{absl}{Absolute}
\baabbrev{abs}{Absolutive}
\baabbrev{acc}{Accusative}
\baabbrev{act}{Active}
\baabbrev{adc}{Adjunct}
\baabbrev{adess}{Adessive}
\baabbrev{adjz}{Adjectivizer}
\baabbrev{adj}{Adjective}
\baabbrev{advz}{Adverbializer}
\baabbrev{adv}{Adverb}
\baabbrev{aff}{Affirmative}
\baabbrev{agn}{Agentive}
\baabbrev{all}{Allative}
\baabbrev{and}{Andative}
\baabbrev{an}{Animate}
\baabbrev{antess}{Antessive}
\baabbrev{aor}{Aorist}
\baabbrev{appl}{Applicative}
\baabbrev{apv}{Antipassive}
\baabbrev{atb}{Autobenefactive}
\baabbrev{atel}{Atelic}
\baabbrev{attr}{Attributive}
\baabbrev{aug}{Augmentative}
\baabbrev{aux}{Auxiliary}
\baabbrev{a}{Agent}
\baabbrev{ben}{Benefactive}
\baabbrev{caus}{Causative}
\baabbrev{cess}{Cessative}
\baabbrev{change}{Change of State}
\baabbrev{cl}{Classifier}
\baabbrev{cmpl}{Completive}
\baabbrev{cmpr}{Comparative}
\baabbrev{col}{Collective}
\baabbrev{com}{Comitative}
\baabbrev{cond}{Conditional}
\baabbrev{conj}{Conjunction}
\baabbrev{conn}{Connective particle}
\baabbrev{conneg}{Connegative}
\baabbrev{con}{Construct}
\baabbrev{cont}{Continous}
\baabbrev{coord}{Coordination}
\baabbrev{cop}{Copula}
\baabbrev{cp}{Complement phrase}
\baabbrev{cvb}{Converb}
\baabbrev{cyc}{Cyclical gender}
\baabbrev{c}{Complementizer}
\baabbrev{dat}{Dative}
\baabbrev{deadline}{Deadline}
\baabbrev{decl}{Declarative}
\baabbrev{def}{Definite}
\baabbrev{deic}{Deictic}
\baabbrev{del}{Delimitative}
\baabbrev{dem}{Demonstrative}
\baabbrev{dep}{Dependent}
\baabbrev{desid}{Desiderative}
\baabbrev{detr}{Detransitivizer}
\baabbrev{det}{Determiner}
\baabbrev{dim}{Diminutive}
\baabbrev{dir}{Direct}
\baabbrev{dis}{Distal/Distant}
\baabbrev{do}{Direct Object}
\baabbrev{dp}{Discourse particle}
\baabbrev{ds}{Different-subject}
\baabbrev{dum}{Dummy pronoun}
\baabbrev{dur}{Durative}
\baabbrev{du}{Dual}
\baabbrev{dvb}{Deverbal}
\baabbrev{ego}{Egophoric}
\baabbrev{emph}{Emphatic}
\baabbrev{erg}{Ergative}
\baabbrev{enc}{Enclitic}
\baabbrev{epis}{Epistemic}
\baabbrev{ess}{Essential} % if someone wants to make ESS essive than imo we should let them and get rid of this unless it comes up again
\baabbrev{ete}{Eternal gender}
\baabbrev{excl}{Exclusive}
\baabbrev{exist}{Existential}
\baabbrev{expl}{Expletive}
\baabbrev{exp}{Experiential; Direct Evidential}
\baabbrev{e}{Edible}
\baabbrev{fam}{Familiar}
\baabbrev{fin}{Finite Verb}
\baabbrev{fish}{Classifier for fish}
\baabbrev{foc}{Focus}
\baabbrev{fol}{Following stance}
\baabbrev{food}{Food}
\baabbrev{for}{Formal}
\baabbrev{frust}{Frustrative}
\baabbrev{fut}{Future}
\baabbrev{f}{Feminine}
\baabbrev{gen}{Genitive}
\baabbrev{ger}{Gerund}
\baabbrev{gno}{Gnomic}
\baabbrev{gnr}{General}
\baabbrev{hab}{Habitual}
\baabbrev{hon}{Honorific}
\baabbrev{hort}{Hortative}
\baabbrev{hyp}{Hypothetical Future}
\baabbrev{h}{Human}
\baabbrev{id}{Ideophone}
\baabbrev{imp}{Imperative}
\baabbrev{inal}{Inalienable Possession}
\baabbrev{inan}{Inanimate}
\baabbrev{inch}{Inchoative}
\baabbrev{incl}{Inclusive}
\baabbrev{inc}{Inceptive}
\baabbrev{indef}{Indefinite}
\baabbrev{ind}{Indicative}
\baabbrev{ine}{Inessive}
\baabbrev{infer}{Inferential}
\baabbrev{inf}{Infinitive}
\baabbrev{instr}{Instrumental}
\baabbrev{ins}{Instantive}
\baabbrev{int}{Intermediate tense}
\baabbrev{inv}{Inverse}
\baabbrev{ipfv}{Imperfective}
\baabbrev{irr}{Irrealis}
\baabbrev{it}{Itive}
\baabbrev{itr}{Iterative}
\baabbrev{itrt}{Intrative}
\baabbrev{juss}{Jussive}
\baabbrev{lat}{Lative}
\baabbrev{lcn}{Location Agreement}
\baabbrev{ld}{Leading Stance}
\baabbrev{lnk}{Linker}
\baabbrev{loc}{Locative}
\baabbrev{l}{Location Anaphor}
\baabbrev{mat}{Material}
\baabbrev{med}{Medial}
\baabbrev{mid}{Middle voice}
\baabbrev{min}{Minimal}
\baabbrev{m}{Masculine}
\baabbrev{name}{Personal Name}
\baabbrev{nec}{Necessitative}
\baabbrev{neg}{Negative}
\baabbrev{next}{Next}
\baabbrev{nfem}{Non-feminine}
\baabbrev{nfi}{Nonfuture imperfective}
\baabbrev{nfor}{Informal}
\baabbrev{nfut}{Non-Future}
\baabbrev{nha}{Non-human animate}
\baabbrev{nmlz}{Nominalizer}
\baabbrev{nml}{Animal Classifier}
\baabbrev{nom}{Nominative}
\baabbrev{npr}{Impersonal}
\baabbrev{npst}{Nonpast}
\baabbrev{nrz}{Non-realized}
\baabbrev{ntr}{Intransitive}
\baabbrev{nvl}{Nonvolitional}
\baabbrev{n}{Neuter}
\baabbrev{oblig}{Obligatory}
\baabbrev{obl}{Oblique}
\baabbrev{obv}{Obviative}
\baabbrev{opt}{Optative}
\baabbrev{org}{Origin}
\baabbrev{orn}{Ornative}
\baabbrev{pa}{Paucal}
\baabbrev{pcp}{Participle}
\baabbrev{perl}{Perlative}
\baabbrev{perm}{Permissive}
\baabbrev{per}{Personal}
\baabbrev{pfv}{Perfective}
\baabbrev{pf}{Patient Focus}
\baabbrev{place}{Place name}
\baabbrev{pl}{Plural}
\baabbrev{pn}{Pronoun, Proper noun}
\baabbrev{pol}{Polite}
\baabbrev{poss}{Possession}
\baabbrev{pos}{Possessive}
\baabbrev{pot}{Potential}
\baabbrev{pp}{Preposition Phrase}
\baabbrev{pret}{Preterite}
\baabbrev{prf}{Perfect}
\baabbrev{priv}{Privative}
\baabbrev{prl}{Prolative}
\baabbrev{prog}{Progressive}
\baabbrev{prop}{Proper article}
\baabbrev{prox}{Proximal}
\baabbrev{prsp}{Prospective}
\baabbrev{prs}{Present}
\baabbrev{pst}{Past}
\baabbrev{psv}{Passive}
\baabbrev{ptv}{Partitive}
\baabbrev{pt}{Plurative}
\baabbrev{punct}{Punctual}
\baabbrev{pv}{Preverb}
\baabbrev{p}{Patient}
\baabbrev{q}{Interrogative}
\baabbrev{quot}{Quotative}
\baabbrev{rdp}{Reduplication}
\baabbrev{rec}{Recent/Near tense}
\baabbrev{refl}{Reflexive}
\baabbrev{rel}{Relative}
\baabbrev{rem}{Remote tense}
\baabbrev{rep}{Reportative}
\baabbrev{res}{Resultative}
\baabbrev[\RR,sort=rr]{r/r}{Reflexive/reciprocal}
\baabbrev{rls}{Realis}
\baabbrev{rvs}{Reversative}
\baabbrev{rz}{Realized}
\baabbrev{sap}{Speech-Act-Participant}
\baabbrev{sbjv}{Subjunctive}
\baabbrev{sca}{Scalar/additive particle}
\baabbrev{sembl}{Semblative}
\baabbrev{seq}{Sequential}
\baabbrev{sg}{Singular}
\baabbrev{sim}{Simple aspect}
\baabbrev{ss}{Same-subject}
\baabbrev{stat}{Stative}
\baabbrev{st}{Singulative}
\baabbrev{sub}{Subordinator}
\baabbrev{supl}{Superlative}
\baabbrev{tel}{Telic}
\baabbrev{tem}{Temporary Gender} % gender is temporary; garlic bread is forever.
\baabbrev{top}{Topic}
\baabbrev{tri}{Trial}
\baabbrev{trn}{Transnumeral}
\baabbrev{trz}{Transitivizer}
\baabbrev{tr}{Transitive}
\baabbrev{trns}{Translative}
\baabbrev{vbz}{Verbializer}
\baabbrev{ven}{Venitive}
\baabbrev{via}{Vialis Argument}
\baabbrev{vid}{Verbal Identifier}
\baabbrev{vis}{Visual}
\baabbrev{voc}{Vocative}
\baabbrev{vol}{Volitional}
\baabbrev{vp}{Verb Phrase}
\baabbrev{v}{Verb}




\begin{document}


\begin{center}
    \includegraphics[width=0.2\textwidth]{segmentslogo.png}
\end{center}

\pagestyle{empty}
\pagenumbering{gobble}

\section*{Instructions}

%Hi! This is the \LaTeX\hspace{.5mm} template for submitting articles for Segments, a Journal of Constructed Languages.

Please copy this project, and only edit the {\color{purplevivid} main.tex} document. Once you are done with your article, share the {\color{purplevivid} edit} link and send it to {\color{purplevivid} segments.journal@gmail.com}.

An article consists of the following, optional items in \highlight{purple}

\begin{enumerate}
    \item The title, subtitle, and author's name
    \item An introduction (250-800 characters incl. spaces)
    \item A phrase, or a handful of sentences, in the conlang, accompanied by its translation
    \item A body (two full pages would be a good minimum)
    \item {\color{purplevivid}A gloss of the phrase(s) from item 3}
    \item {\color{purplevivid}A few pictures, graphs, tables, or other useful graphics}
\end{enumerate}

Highlight important information using the command \textbackslash highlight: \highlight{highlighted text here}

\subsection*{Title, subtitle}

The title of your article will generally include the name of your conlang or project. The subtitle can be humorous, or descriptive, or somewhere in between!

Here are some examples of perfectly decent title -- subtitle combinations:

\begin{itemize}
    \item Mwaneḷe -- Ụṇḍẹṛḍọṭṣ
    \item ǂa ɳṵĩ -- A naturalistic click language
    \item Wistanian, a phonology of -- An odyssey into the sounds of Wistania
    \item An introduction to Valdean -- A written language
\end{itemize}

\subsection*{Graphics, tables}

The inclusion of graphics is there so you can indicate where the images you want to include in your article are best placed, according to you. We will try our best to accomodate your choices, however please be aware that differences may arise in the final product to facilitate laying it out. We will keep these differences as small as possible, and will send you a copy of your article for review before publication.

You can include them with the command \verb|\includegraphics[]{name_of_image}| (see example above in this document's code with the logo).

\subsection*{Notes, comments, and asides}

You can add notes and asides as footnotes to this document. We will integrate them into the article.\footnote{Use \textbackslash footnote\{\} to include a footnote!}




\newpage




\section*{Styling guide: Format \& examples}

You can copy code from this section and alter it to fit your needs.

\subsection*{Paragraph styles}

\subsubsection*{Conlang, translations}

Words, or morphemes, of the conlang should be \textbf{bolded} if they are inside of an English sentence. 

English translations should be in \textit{italics}, and

\begin{itemize}
    \item surrounded by \highlight{`single quotemarks'} if it is a single word, or a definition;
    \item surrounded by \highlight{``double quotemarks"} inside examples and glosses.
\end{itemize}

IPA transcriptions should be left in plain, normal text, unless specific phonemes are talked about within an English sentence, in which case they must be in \textit{italics}, and can optionally be \highlight{highlighted in purple}.

Here is an example: \\
The verb \textbf{puqwe} /pɯqʷɛ/ \textit{`to clean up, to tidy up, to arrange, to put in order'} is used with bananas. here, the phoneme \highlight{\textit{qʷ}} becomes [t͡l] through assimilation of the PoA of /j/ which is not present in this word at all, and this sentence makes no sense.

\subsubsection*{Conscripts}

The best way to include a conscript is to send, along with your article, a high resolution .png, or a .svg of it, along with all relevant instructions for its placement.

\subsubsection{Custom fonts}

If your conlang makes use of a writing system for which several fonts are available, please provide us with a .ttf or .otf file of the one you would like us to use, \textbf{after making sure that font is under a license that would allow for a third party publication to make use of it.}

Your TeX file should come with the necessary packages and commands already implemented at the top of \textbf{main.tex}, right after line 3, which inputs our \textbf{specifications.tex} file.

If no such font is available, a high resolution .png, or a .svg of it would be ideal.

\subsubsection{Additional specifications}

Latin locutions, such as \textit{etc}, \textit{e.g.}, \textit{i.e.} and others should be in \textit{italics}.

\subsection*{Tables}

Headers for both columns and rows are in bold. Row headers are aligned to the left, and everything else to the center.

The only two lines drawn are right next to the headers, and run across the entirety of the table.

\begin{table}[H]
	\centering
	\begin{tabu}{$>{\bfseries}l|^c^c^c}     % ^ before a column whose header should be bold
	\rowstyle{\bfseries}                    % The next row will be bold
		& Labial & Alveolar & Dorsal \\
		\hline
		Nasal       & m     		   & n        		   & q      \\
		Stop        & b     		   & t d      		   & k g    \\
		Fricative   & f     		   & s x     		   & c h    \\
		Trill       &        		   & r       		   &        \\
		Approximant & w    		   & l      		   & j      
	\end{tabu}
	\caption{A single table}
	\label{cons-inv}
\end{table}

\begin{figure}[H]

\begin{multicols}{2}
\centering

    \begin{tabu}{$>{\bfseries}l|^c^c}
	\rowstyle{\bfseries}
    & Front & Back \\
    \hline
    High & i & u \\
    Mid & ɛ & ɔ\\
    Low & a & ɒ
    \end{tabu}

    \begin{tabu}{$>{\bfseries}l|^c^c}
	\rowstyle{\bfseries}
    & Unrounded & Rounded \\
    \hline
    High & i & u \\
    Mid & ɛ & ɔ\\
    Low & a & ɒ
    \end{tabu}

\end{multicols}

\caption{Here are two tables, on the same level}
\end{figure}
\clearpage 
\subsection*{Vowel charts}

Vowel charts can be constructed using the \highlight{\texttt{tikzvowel}} package. To create a vowel chart, begin a \texttt{vowel} environment. Here is an example vowel chart. You can modify the size of the trapezoid by adjusting the \textbf{scale} parameter.

\begin{figure}[H]
  \centering
  \begin{vowel}[scale=0.75]
    \vpoint{0}{0}{i}
    \vpoint{0}{2}{u}
    \vpoint{1.5}{1}{ə}
    \vpoint{3}{1}{a}
    \vblob{-0.1, 2.1}{1.2, 2.1}{1.2, 1.7}{-0.1, 1.7}
    \varrow{a}{i}
  \end{vowel}
  \caption{Phonemic Vowel Inventory}
  \label{table:vowel_phonemes}
\end{figure}
\begin{verbbox}
  \begin{vowel}[scale=0.75]
    \vpoint{0}{0}{i}
    \vpoint{0}{2}{u}
    \vpoint{1.5}{1}{ə}
    \vpoint{3}{1}{a}
    \vblob{-0.1, 2.1}{1.2, 2.1}{1.2, 1.7}{-0.1, 1.7}
    \varrow{a}{i}
  \end{vowel}
\end{verbbox}
\fbox{\theverbbox} 

Add vowels using the command \texttt{\textbackslash vpoint\{y\}\{x\}\{V\}}, where \texttt{x} is the horizontal position of the vowel, \texttt{y} is the vertical position of the vowel, and \texttt{V} is the vowel letter you want to add. The origin is set at the top left corner. If \textbf{scale} is set to 1, then each horizontal or vertical line is one unit. For example in figure \ref{table:vowel_phonemes}, \textbf{a} is inserted three columns down and one row to the right, using the command \texttt{\textbackslash vpoint\{3\}\{1\}\{a\}}. The x and y positions don't have to be integers: to place the schwa between the mid-closed line and the mid-open line, we used y=1.5.

In addition to adding vowel letters, the \texttt{tikzvowel} package can add arrows to show diphthongs or vowel shifts, as well as outlines to show ranges within the vowel chart. Arrows can be added using the command \texttt{\textbackslash varrow\{a\}\{b\}}, which draws an arrow from vowel a to vowel b. Here, the arrow from \textbf{a} to \textbf{i} is drawn with \texttt{\textbackslash varrow\{a\}\{i\}}. 

Dotted outlines can be drawn with the \texttt{\textbackslash vblob} command, which takes a series of coordinates and draws a curved shape with those coordinates as rough corners. The dotted blob showing the vowel space occupied by the phoneme /u/ in figure \ref{table:vowel_phonemes} is drawn with the command \texttt{\textbackslash vblob\{-0.1, 2.1\}\{1.2, 2.1\}\{1.2, 1.7\}\{-0.1, 1.7\}}. 

Beware that this package gives coordinates in (y, x) and uses the top left as the origin point (unlike what they tend to teach in school!). Other than that point, this is an easy and versatile tool to draw vowel diagrams!

\newpage
\section*{Using Baarux}
\counterwithout{exesi}{section} % needed to get baarux to number the way I want outside of chapters

In Segments, we use the package \highlight{\texttt{baarux}} by community member \highlight{Akam Chinjir} to typeset examples. The current version as of writing is \highlight{0.9.9}, which we've included in this template. As Akam pushes new versions, we'll update our documents including this guide. \texttt{baarux} is a very powerful package, so we won't go into everything it can do, but here's an introduction to its functionalities. % link to akam's documentation as extra reference when it's done

The core of \texttt{baarux} is the \highlight{examples} environment, which creates numbered examples. Declare a new example within the examples environment using \texttt{\textbackslash ex}. You can nest examples environments to make subexamples (up to four levels deep, but if you need more than that, then stop and think about your life choices). You can use the \texttt{\textbackslash label} and \texttt{\textbackslash ref} commands to label examples and refer back to them by number, like example \ref{ex:protest}.

I'll follow each example with the code used to make it, so that you can get a sense of how \texttt{baarux} is used.

\begin{examples}
    \ex This is a test example
    \ex And these are some test subexamples:
    \begin{examples}
        \ex Test
        \ex Test
        \ex Protest \label{ex:protest}
    \end{examples}
\end{examples}
\begin{verbbox}
\begin{examples}
    \ex This is a test example
    \ex And these are some test subexamples:
    \begin{examples}
        \ex Test
        \ex Test
        \ex Protest \label{ex:protest}
    \end{examples}
\end{examples}
\end{verbbox}
\fbox{\theverbbox} 

Within the examples environment, \texttt{baarux} defines several line types. Here are the predefined plain line types:
\begin{itemize}
    \item \highlight{\texttt{\textbackslash preamble}} gives an unformatted line
    \item \highlight{\texttt{\textbackslash script}} is bolded and is meant for the in-language text of the example
    \item \highlight{\texttt{\textbackslash tr}} is automatically surrounded by quotation marks and is meant for the translation
    \item \highlight{\texttt{\textbackslash context}} prints the line preceded by `Context: '
    \item \highlight{\texttt{\textbackslash alt}} prints the line inside quotation marks and preceded by \textit{`Or: '} and is used to introduce alternate translations
    \item \highlight{\texttt{\textbackslash intended}} prints the line inside quotation marks and preceded by \textit{`Intended: '} and can be used to give the intended meaning of an ungrammatical or questionable sentence 
    \item \highlight{\texttt{\textbackslash not}} prints the line inside quotation marks and preceded by \textit{`Not: '} and can be used to give examples of what a sentence \emph{doesn't} mean to contrast with what it does mean
\end{itemize} 
There are also two predefined line types that give right-aligned comments: \highlight{\texttt{\textbackslash lect}} prints a plain right-aligned comment and is meant to give the name of the language or variety used in the example, while \highlight{\texttt{\textbackslash source}} prints a right-aligned comment in parentheses and is meant to show where the text in the example is coming from. (These are the intended uses, but of course you can be creative! Reach out if you're interested in learning how to define your own line types.)

Example \ref{ex:seoina} shows a sentence written in the conlang Seoina taken from the \href{https://www.reddit.com/r/conlangs/comments/haaf0c/1278th_just_used_5_minutes_of_your_day/} {1278\textsuperscript{th} Just Used 5 Minutes of your Day Challenge} from r/conlangs. 

\begin{examples}
    \ex \label{ex:seoina}
    \lect Seoina
    \script Sa kel si la deol, aloi la kipia nolra peu deoi.
    \tr How you treat yourself, that's how people will think to treat you.
    \alt People will think to treat you however you treat yourself.
    \source \textsc{5moyd} \#1278
\end{examples}

\begin{verbbox}
\begin{examples}
    \ex \label{ex:seoina}
    \lect Seoina
    \script Sa kel si la deol, aloi la kipia nolra peu deoi.
    \tr How you treat yourself, that's how people will think to treat you.
    \alt People will think to treat you however you treat yourself.
    \source \textsc{5moyd} \#1278
\end{examples}
\end{verbbox}
\fbox{\theverbbox}

You can also easily do glossed example sentences with \texttt{baarux}. In addition to plain lines and right-aligned comments, the package defines three types of lines that automatically align for glossing:

\begin{itemize}
    \item \highlight{\texttt{\textbackslash words}} is automatically bolded and meant for words in the language
    \item \highlight{\texttt{\textbackslash bits}} is plain and meant for words, morphemes, affixes, clitics...you know, `bits'
    \item \highlight{\texttt{\textbackslash gloss}} is italicized except for glossing abbreviations, which are automatically detected and rendered in upright smallcaps--more on that later
\end{itemize}

The \texttt{baarux} package comes with a companion \texttt{baabbrevs} which is used to define glossing abbreviations. Abbreviations are defined using the command \texttt{\textbackslash baabbrev}, which takes two arguments: the abbreviation in all lowercase followed optionally by the full term that the abbreviation is short for. You can see examples of this in the \highlight{abbreviations.tex} file under the Required folder here. Please use glossing terms that are already present in that list whenever possible; if you have additional glossing abbreviations needed that are \textbf{not} in that list, please add them to the top of your \highlight{main.tex} file so they can be properly added when Segments is compiled completely.

To make a glossed example, add your words or bits, separated by hyphens or equals signs wherever you want there to be junctures. Then add a gloss row with \textbf{only spaces} between the bits of gloss, and with defined abbreviations in all caps. \texttt{baarux} will automatically break them up and align them, add hyphens and equals signs to the gloss row corresponding to whatever's in the bits row, and turn abbreviations into \textsc{small caps}. You can use parentheses, colons and periods in the gloss line without messing things up. You can change the alignment of individual bits by using spaces to separate juncture markers in the bits line.

\begin{examples}
    \ex \label{ex:mwanele}
    \lect Mwaneḷe
    \script Mwana xalolo xo tetesi exeŋi ekwulife.
    \bits Mwana xalolo =xo ta- e- tesi e- xeŋi e- kwu- life
    \gloss NAME fear:NFI DP C APV exceed APV be.below APV VEN arrive
    \tr Mwana is afraid she arrived too late.
    \source \textsc{5moyd} \#1270
\end{examples}

\begin{verbbox}
\begin{examples}
    \ex \label{ex:mwanele}
    \lect Mwaneḷe
    \script Mwana xalolo xo tetesi exeŋi ekwulife.
    \bits Mwana xalolo =xo ta- e- tesi e- xeŋi e- kwu- life
    \gloss NAME fear:NFI DP CMP APV exceed APV be.below APV VEN arrive
    \tr Mwana is afraid she arrived too late.
    \source \textsc{5moyd} \#1270
\end{examples}
\end{verbbox}
\fbox{\theverbbox}

Sometimes it can be useful to have items in glossed lines that extend across multiple columns. You can do this easily with the \highlight{\texttt{\textbackslash MC}} command. If you put \texttt{\textbackslash MC} followed by a number \textit{n}, that tells \texttt{baarux} that the next word should occupy \textit{n} cells. 

On the other hand, sometimes you want multiple words in a single cell or column. If you want to include a string with a space in it, then you can surround it by brackets \{like this\} to tell \texttt{baarux} to treat it as one unit, instead of treating the space as a word/morpheme break.

\begin{examples}
    \ex
    \lect 3eyri
    \words Čelt \MC4 tanolhka \MC2 piko mi \MC3 tagökalh?
    \bits čelt t -ano  -lh =ga pik =o mi t -gök -lh
    \gloss man 3ACC sleep 3NOM in boy NOM what 3ACC eat 3NOM
    \gloss man \MC4 {while he slept} \MC2 {the boy} what \MC3 {he ate it}
    \tr What did the boy eat while the man was sleeping?
    \source \textsc{5moyd} \#1179
\end{examples}

\begin{verbbox}
\begin{examples}
    \ex
    \lect 3eyri
    \words Čelt \MC4 tanolhka \MC2 piko mi \MC3 tagökalh?
    \bits čelt t -ano  -lh =ga pik =o mi t -gök -lh
    \gloss man 3ACC sleep 3NOM in boy NOM what 3ACC eat 3NOM
    \gloss man \MC4 {while he slept} \MC2 {the boy} what \MC3 {he ate it}
    \tr What did the boy eat while the man was sleeping?
    \source \textsc{5moyd} \#1179
\end{examples}
\end{verbbox}
\fbox{\theverbbox}

You can already see that \texttt{baarux} is a very powerful package, and this is just the tip of the iceberg! If there's something you want to do with your examples, take a look at Akam's documentation or ask around. Chances are you can find a way.


\newpage
\mbox{}
\newpage

\pagenumbering{arabic}
\pagestyle{fancy} % Comment this to remove the help pages

%------------------------------------------------------
%   ↓   INTRODUCTION    ↓
%------------------------------------------------------

\chapter[Randomizing Language Features][Tamgwa]{Randomizing Language Features in Tamgwa}

\byline{armytag} % add your name as you wish to be credited
\articlesub{Overcoming indecision, discovering new possibilities} % use this command to insert a subtitle

\thispagestyle{fancy}
\BgUsetrue

%------------------------------------------------------
%   ↓   ARTICLE BODY    ↓
%------------------------------------------------------

Insert an introduction here. 250-800 characters, including spaces.

\section*{Why use randomization?} % * for unnumbered section

\lipsum[1]

\subsection*{A basic example}

\lipsum[2]

\section*{Laying the foundation} % * for unnumbered section

\lipsum[4]

\section*{Personal disagreement} % * for unnumbered section

\lipsum[4]

\begin{examples}
    \ex
    \lect Sorani Kurdish
    \words \MC2 dat \MC2 benim
    \bits da =t beni -m
    \gloss IPFV OBL.2SG see.IPFV 1SG
    \tr I see you.
    \ex
    \words \MC3 benemeet
    \bits beni =m -eet
    \gloss see.PFV OBL.1SG 2SG
    \tr I saw you.
    \ex
    \words \MC3 benetm
    \bits bene =t -m
    \gloss see.PFV OBL.2SG 1SG
    \tr You saw me.
    \source Gharib \& Pye 2023
\end{examples}

Note that Sorani Kurdish changes whether the agent (A) or patient (P) of the verb is marked with the oblique case depending on whether the verb is perfective or imperfective.  For our purposes, it is sufficient to see that examples (1) and (3) show the OBL.2SG clitic attaching to the aspect marker when it is present.  We can use the same system in Tamgwa to satisfy our conflicting features.

(see Gharib \& Pye 2023)\footnote{Gharib, H., \& Pye, C. (2023) The clitic status of person markers in Sorani Kurdish. Kansas Working Papers in Linguistics, 39, 57-65. https://doi.org/10.17161/1808.27692}

\begin{table}[H]
	\centering
	\begin{tabu}{$>{\bfseries}l|^c^c^c^c^c}
	\rowstyle{\bfseries}
        & NOM & Affix & ACC & Clitic & GEN \\
		\hline
        1SG     & hnja      & ki    & gya       & hnja  & hma    \\
        2SG     & lam       & man   & ba        & be    & pu     \\
        3SG     & le        & na    & bwa       & nji   & pe     \\
        1PL.INC & ga        & fe    & hmuhl     & pix   & re     \\
        1PL.EXC & pwa       & tenj  & hla       & le    & pu     \\
        2PL     & bihl      & tu    & bya       & bes   & hnjanj \\
        3PL     & xa        & ti    & pawx      & njar  & bwa    \\
	\end{tabu}
	\caption{Random pronoun table}
	\label{rand-pronoun}
\end{table}

From this randomized starting point, we can look for patterns and rearrange the table until we're satisfied.  Personally, I latched onto the two \textbf{hnja} in the 1SG row, so I moved all the words beginning with \textbf{hnj} or \textbf{nj} into that row.  This continued with other patters, such as voiceless labials for 2SG and closed syllables for GEN.  The final table is printed below, with obligatory plural marker \textbf{tyan} added to NOM and ACC columns.

\begin{table}[H]
	\centering
	\begin{tabu}{$>{\bfseries}l|^c^c^c^c^c}
	\rowstyle{\bfseries}
        & NOM & Affix & ACC & Clitic & GEN \\
		\hline
        1SG     & hnja      & -nji  & hnja      & =njar & hnjanj \\
        2SG     & fe        & -pe   & pawx      & =pwa  & pix    \\
        3SG     & bes       & -hla  & hmuhl     & =hma  & man    \\
        1PL.INC & tu tyan   & -ti   & re tyan   & =na   & tenj   \\
        1PL.EXC & xa tyan   & -ki   & gya tyan  & =ga   & bya    \\
        2PL     & bwa tyan  & -be   & bwa tyan  & =ba   & bihl   \\
        3PL     & pu tyan   & -le   & pu tyan   & =le   & lam    \\
	\end{tabu}
	\caption{Rearranged pronoun table}
	\label{arr-pronoun}
\end{table}

\begin{examples}
    \ex
    \lect Tamgwa
    \words \MC3 kyamunjipwa
    \bits kyamu -nji =pwa
    \gloss see 1SG.NOM 2SG.ACC
    \tr I saw you.
    \ex
    \words \MC2 kwapwa \MC2 kyamunji
    \bits kwa =pwa kyamu -nji
    \gloss IPFV 2SG.ACC see 1SG.NOM
    \tr I see you.
\end{examples}

\section*{Resolving the tense situation} % * for unnumbered section

\lipsum[6]

\section*{Other randomization ideas} % * for unnumbered section

\lipsum[7]

\section*{Conclusion} % * for unnumbered section

\lipsum[8]

\end{document}

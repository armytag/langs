
\begin{center}
    \includegraphics[width=0.2\textwidth]{segmentslogo.png}
\end{center}

\pagestyle{empty}
\pagenumbering{gobble}

\section*{Instructions}

%Hi! This is the \LaTeX\hspace{.5mm} template for submitting articles for Segments, a Journal of Constructed Languages.

Please copy this project, and only edit the {\color{purplevivid} main.tex} document. Once you are done with your article, share the {\color{purplevivid} edit} link and send it to {\color{purplevivid} segments.journal@gmail.com}.

An article consists of the following, optional items in \highlight{purple}

\begin{enumerate}
    \item The title, subtitle, and author's name
    \item An introduction (250-800 characters incl. spaces)
    \item A phrase, or a handful of sentences, in the conlang, accompanied by its translation
    \item A body (two full pages would be a good minimum)
    \item {\color{purplevivid}A gloss of the phrase(s) from item 3}
    \item {\color{purplevivid}A few pictures, graphs, tables, or other useful graphics}
\end{enumerate}

Highlight important information using the command \textbackslash highlight: \highlight{highlighted text here}

\subsection*{Title, subtitle}

The title of your article will generally include the name of your conlang or project. The subtitle can be humorous, or descriptive, or somewhere in between!

Here are some examples of perfectly decent title -- subtitle combinations:

\begin{itemize}
    \item Mwaneḷe -- Ụṇḍẹṛḍọṭṣ
    \item ǂa ɳṵĩ -- A naturalistic click language
    \item Wistanian, a phonology of -- An odyssey into the sounds of Wistania
    \item An introduction to Valdean -- A written language
\end{itemize}

\subsection*{Graphics, tables}

The inclusion of graphics is there so you can indicate where the images you want to include in your article are best placed, according to you. We will try our best to accomodate your choices, however please be aware that differences may arise in the final product to facilitate laying it out. We will keep these differences as small as possible, and will send you a copy of your article for review before publication.

You can include them with the command \verb|\includegraphics[]{name_of_image}| (see example above in this document's code with the logo).

\subsection*{Notes, comments, and asides}

You can add notes and asides as footnotes to this document. We will integrate them into the article.\footnote{Use \textbackslash footnote\{\} to include a footnote!}




\newpage




\section*{Styling guide: Format \& examples}

You can copy code from this section and alter it to fit your needs.

\subsection*{Paragraph styles}

\subsubsection*{Conlang, translations}

Words, or morphemes, of the conlang should be \textbf{bolded} if they are inside of an English sentence. 

English translations should be in \textit{italics}, and

\begin{itemize}
    \item surrounded by \highlight{`single quotemarks'} if it is a single word, or a definition;
    \item surrounded by \highlight{``double quotemarks"} inside examples and glosses.
\end{itemize}

IPA transcriptions should be left in plain, normal text, unless specific phonemes are talked about within an English sentence, in which case they must be in \textit{italics}, and can optionally be \highlight{highlighted in purple}.

Here is an example: \\
The verb \textbf{puqwe} /pɯqʷɛ/ \textit{`to clean up, to tidy up, to arrange, to put in order'} is used with bananas. here, the phoneme \highlight{\textit{qʷ}} becomes [t͡l] through assimilation of the PoA of /j/ which is not present in this word at all, and this sentence makes no sense.

\subsubsection*{Conscripts}

The best way to include a conscript is to send, along with your article, a high resolution .png, or a .svg of it, along with all relevant instructions for its placement.

\subsubsection{Custom fonts}

If your conlang makes use of a writing system for which several fonts are available, please provide us with a .ttf or .otf file of the one you would like us to use, \textbf{after making sure that font is under a license that would allow for a third party publication to make use of it.}

Your TeX file should come with the necessary packages and commands already implemented at the top of \textbf{main.tex}, right after line 3, which inputs our \textbf{specifications.tex} file.

If no such font is available, a high resolution .png, or a .svg of it would be ideal.

\subsubsection{Additional specifications}

Latin locutions, such as \textit{etc}, \textit{e.g.}, \textit{i.e.} and others should be in \textit{italics}.

\subsection*{Tables}

Headers for both columns and rows are in bold. Row headers are aligned to the left, and everything else to the center.

The only two lines drawn are right next to the headers, and run across the entirety of the table.

\begin{table}[H]
	\centering
	\begin{tabu}{$>{\bfseries}l|^c^c^c}     % ^ before a column whose header should be bold
	\rowstyle{\bfseries}                    % The next row will be bold
		& Labial & Alveolar & Dorsal \\
		\hline
		Nasal       & m     		   & n        		   & q      \\
		Stop        & b     		   & t d      		   & k g    \\
		Fricative   & f     		   & s x     		   & c h    \\
		Trill       &        		   & r       		   &        \\
		Approximant & w    		   & l      		   & j      
	\end{tabu}
	\caption{A single table}
	\label{cons-inv}
\end{table}

\begin{figure}[H]

\begin{multicols}{2}
\centering

    \begin{tabu}{$>{\bfseries}l|^c^c}
	\rowstyle{\bfseries}
    & Front & Back \\
    \hline
    High & i & u \\
    Mid & ɛ & ɔ\\
    Low & a & ɒ
    \end{tabu}

    \begin{tabu}{$>{\bfseries}l|^c^c}
	\rowstyle{\bfseries}
    & Unrounded & Rounded \\
    \hline
    High & i & u \\
    Mid & ɛ & ɔ\\
    Low & a & ɒ
    \end{tabu}

\end{multicols}

\caption{Here are two tables, on the same level}
\end{figure}
\clearpage 
\subsection*{Vowel charts}

Vowel charts can be constructed using the \highlight{\texttt{tikzvowel}} package. To create a vowel chart, begin a \texttt{vowel} environment. Here is an example vowel chart. You can modify the size of the trapezoid by adjusting the \textbf{scale} parameter.

\begin{figure}[H]
  \centering
  \begin{vowel}[scale=0.75]
    \vpoint{0}{0}{i}
    \vpoint{0}{2}{u}
    \vpoint{1.5}{1}{ə}
    \vpoint{3}{1}{a}
    \vblob{-0.1, 2.1}{1.2, 2.1}{1.2, 1.7}{-0.1, 1.7}
    \varrow{a}{i}
  \end{vowel}
  \caption{Phonemic Vowel Inventory}
  \label{table:vowel_phonemes}
\end{figure}
\begin{verbbox}
  \begin{vowel}[scale=0.75]
    \vpoint{0}{0}{i}
    \vpoint{0}{2}{u}
    \vpoint{1.5}{1}{ə}
    \vpoint{3}{1}{a}
    \vblob{-0.1, 2.1}{1.2, 2.1}{1.2, 1.7}{-0.1, 1.7}
    \varrow{a}{i}
  \end{vowel}
\end{verbbox}
\fbox{\theverbbox} 

Add vowels using the command \texttt{\textbackslash vpoint\{y\}\{x\}\{V\}}, where \texttt{x} is the horizontal position of the vowel, \texttt{y} is the vertical position of the vowel, and \texttt{V} is the vowel letter you want to add. The origin is set at the top left corner. If \textbf{scale} is set to 1, then each horizontal or vertical line is one unit. For example in figure \ref{table:vowel_phonemes}, \textbf{a} is inserted three columns down and one row to the right, using the command \texttt{\textbackslash vpoint\{3\}\{1\}\{a\}}. The x and y positions don't have to be integers: to place the schwa between the mid-closed line and the mid-open line, we used y=1.5.

In addition to adding vowel letters, the \texttt{tikzvowel} package can add arrows to show diphthongs or vowel shifts, as well as outlines to show ranges within the vowel chart. Arrows can be added using the command \texttt{\textbackslash varrow\{a\}\{b\}}, which draws an arrow from vowel a to vowel b. Here, the arrow from \textbf{a} to \textbf{i} is drawn with \texttt{\textbackslash varrow\{a\}\{i\}}. 

Dotted outlines can be drawn with the \texttt{\textbackslash vblob} command, which takes a series of coordinates and draws a curved shape with those coordinates as rough corners. The dotted blob showing the vowel space occupied by the phoneme /u/ in figure \ref{table:vowel_phonemes} is drawn with the command \texttt{\textbackslash vblob\{-0.1, 2.1\}\{1.2, 2.1\}\{1.2, 1.7\}\{-0.1, 1.7\}}. 

Beware that this package gives coordinates in (y, x) and uses the top left as the origin point (unlike what they tend to teach in school!). Other than that point, this is an easy and versatile tool to draw vowel diagrams!

\newpage
\section*{Using Baarux}
\counterwithout{exesi}{section} % needed to get baarux to number the way I want outside of chapters

In Segments, we use the package \highlight{\texttt{baarux}} by community member \highlight{Akam Chinjir} to typeset examples. The current version as of writing is \highlight{0.9.9}, which we've included in this template. As Akam pushes new versions, we'll update our documents including this guide. \texttt{baarux} is a very powerful package, so we won't go into everything it can do, but here's an introduction to its functionalities. % link to akam's documentation as extra reference when it's done

The core of \texttt{baarux} is the \highlight{examples} environment, which creates numbered examples. Declare a new example within the examples environment using \texttt{\textbackslash ex}. You can nest examples environments to make subexamples (up to four levels deep, but if you need more than that, then stop and think about your life choices). You can use the \texttt{\textbackslash label} and \texttt{\textbackslash ref} commands to label examples and refer back to them by number, like example \ref{ex:protest}.

I'll follow each example with the code used to make it, so that you can get a sense of how \texttt{baarux} is used.

\begin{examples}
    \ex This is a test example
    \ex And these are some test subexamples:
    \begin{examples}
        \ex Test
        \ex Test
        \ex Protest \label{ex:protest}
    \end{examples}
\end{examples}
\begin{verbbox}
\begin{examples}
    \ex This is a test example
    \ex And these are some test subexamples:
    \begin{examples}
        \ex Test
        \ex Test
        \ex Protest \label{ex:protest}
    \end{examples}
\end{examples}
\end{verbbox}
\fbox{\theverbbox} 

Within the examples environment, \texttt{baarux} defines several line types. Here are the predefined plain line types:
\begin{itemize}
    \item \highlight{\texttt{\textbackslash preamble}} gives an unformatted line
    \item \highlight{\texttt{\textbackslash script}} is bolded and is meant for the in-language text of the example
    \item \highlight{\texttt{\textbackslash tr}} is automatically surrounded by quotation marks and is meant for the translation
    \item \highlight{\texttt{\textbackslash context}} prints the line preceded by `Context: '
    \item \highlight{\texttt{\textbackslash alt}} prints the line inside quotation marks and preceded by \textit{`Or: '} and is used to introduce alternate translations
    \item \highlight{\texttt{\textbackslash intended}} prints the line inside quotation marks and preceded by \textit{`Intended: '} and can be used to give the intended meaning of an ungrammatical or questionable sentence 
    \item \highlight{\texttt{\textbackslash not}} prints the line inside quotation marks and preceded by \textit{`Not: '} and can be used to give examples of what a sentence \emph{doesn't} mean to contrast with what it does mean
\end{itemize} 
There are also two predefined line types that give right-aligned comments: \highlight{\texttt{\textbackslash lect}} prints a plain right-aligned comment and is meant to give the name of the language or variety used in the example, while \highlight{\texttt{\textbackslash source}} prints a right-aligned comment in parentheses and is meant to show where the text in the example is coming from. (These are the intended uses, but of course you can be creative! Reach out if you're interested in learning how to define your own line types.)

Example \ref{ex:seoina} shows a sentence written in the conlang Seoina taken from the \href{https://www.reddit.com/r/conlangs/comments/haaf0c/1278th_just_used_5_minutes_of_your_day/} {1278\textsuperscript{th} Just Used 5 Minutes of your Day Challenge} from r/conlangs. 

\begin{examples}
    \ex \label{ex:seoina}
    \lect Seoina
    \script Sa kel si la deol, aloi la kipia nolra peu deoi.
    \tr How you treat yourself, that's how people will think to treat you.
    \alt People will think to treat you however you treat yourself.
    \source \textsc{5moyd} \#1278
\end{examples}

\begin{verbbox}
\begin{examples}
    \ex \label{ex:seoina}
    \lect Seoina
    \script Sa kel si la deol, aloi la kipia nolra peu deoi.
    \tr How you treat yourself, that's how people will think to treat you.
    \alt People will think to treat you however you treat yourself.
    \source \textsc{5moyd} \#1278
\end{examples}
\end{verbbox}
\fbox{\theverbbox}

You can also easily do glossed example sentences with \texttt{baarux}. In addition to plain lines and right-aligned comments, the package defines three types of lines that automatically align for glossing:

\begin{itemize}
    \item \highlight{\texttt{\textbackslash words}} is automatically bolded and meant for words in the language
    \item \highlight{\texttt{\textbackslash bits}} is plain and meant for words, morphemes, affixes, clitics...you know, `bits'
    \item \highlight{\texttt{\textbackslash gloss}} is italicized except for glossing abbreviations, which are automatically detected and rendered in upright smallcaps--more on that later
\end{itemize}

The \texttt{baarux} package comes with a companion \texttt{baabbrevs} which is used to define glossing abbreviations. Abbreviations are defined using the command \texttt{\textbackslash baabbrev}, which takes two arguments: the abbreviation in all lowercase followed optionally by the full term that the abbreviation is short for. You can see examples of this in the \highlight{abbreviations.tex} file under the Required folder here. Please use glossing terms that are already present in that list whenever possible; if you have additional glossing abbreviations needed that are \textbf{not} in that list, please add them to the top of your \highlight{main.tex} file so they can be properly added when Segments is compiled completely.

To make a glossed example, add your words or bits, separated by hyphens or equals signs wherever you want there to be junctures. Then add a gloss row with \textbf{only spaces} between the bits of gloss, and with defined abbreviations in all caps. \texttt{baarux} will automatically break them up and align them, add hyphens and equals signs to the gloss row corresponding to whatever's in the bits row, and turn abbreviations into \textsc{small caps}. You can use parentheses, colons and periods in the gloss line without messing things up. You can change the alignment of individual bits by using spaces to separate juncture markers in the bits line.

\begin{examples}
    \ex \label{ex:mwanele}
    \lect Mwaneḷe
    \script Mwana xalolo xo tetesi exeŋi ekwulife.
    \bits Mwana xalolo =xo ta- e- tesi e- xeŋi e- kwu- life
    \gloss NAME fear:NFI DP C APV exceed APV be.below APV VEN arrive
    \tr Mwana is afraid she arrived too late.
    \source \textsc{5moyd} \#1270
\end{examples}

\begin{verbbox}
\begin{examples}
    \ex \label{ex:mwanele}
    \lect Mwaneḷe
    \script Mwana xalolo xo tetesi exeŋi ekwulife.
    \bits Mwana xalolo =xo ta- e- tesi e- xeŋi e- kwu- life
    \gloss NAME fear:NFI DP CMP APV exceed APV be.below APV VEN arrive
    \tr Mwana is afraid she arrived too late.
    \source \textsc{5moyd} \#1270
\end{examples}
\end{verbbox}
\fbox{\theverbbox}

Sometimes it can be useful to have items in glossed lines that extend across multiple columns. You can do this easily with the \highlight{\texttt{\textbackslash MC}} command. If you put \texttt{\textbackslash MC} followed by a number \textit{n}, that tells \texttt{baarux} that the next word should occupy \textit{n} cells. 

On the other hand, sometimes you want multiple words in a single cell or column. If you want to include a string with a space in it, then you can surround it by brackets \{like this\} to tell \texttt{baarux} to treat it as one unit, instead of treating the space as a word/morpheme break.

\begin{examples}
    \ex
    \lect 3eyri
    \words Čelt \MC4 tanolhka \MC2 piko mi \MC3 tagökalh?
    \bits čelt t -ano  -lh =ga pik =o mi t -gök -lh
    \gloss man 3ACC sleep 3NOM in boy NOM what 3ACC eat 3NOM
    \gloss man \MC4 {while he slept} \MC2 {the boy} what \MC3 {he ate it}
    \tr What did the boy eat while the man was sleeping?
    \source \textsc{5moyd} \#1179
\end{examples}

\begin{verbbox}
\begin{examples}
    \ex
    \lect 3eyri
    \words Čelt \MC4 tanolhka \MC2 piko mi \MC3 tagökalh?
    \bits čelt t -ano  -lh =ga pik =o mi t -gök -lh
    \gloss man 3ACC sleep 3NOM in boy NOM what 3ACC eat 3NOM
    \gloss man \MC4 {while he slept} \MC2 {the boy} what \MC3 {he ate it}
    \tr What did the boy eat while the man was sleeping?
    \source \textsc{5moyd} \#1179
\end{examples}
\end{verbbox}
\fbox{\theverbbox}

You can already see that \texttt{baarux} is a very powerful package, and this is just the tip of the iceberg! If there's something you want to do with your examples, take a look at Akam's documentation or ask around. Chances are you can find a way.


\newpage
\mbox{}
\newpage

\pagenumbering{arabic}
\pagestyle{fancy}
%------------------------------------------------------
%   PACKAGES SET-UP
%------------------------------------------------------

%---------------
%   PAGE
%---------------

\usepackage{geometry}   % Required for adjusting page dimensions and margins

\geometry{
	paper=a4paper,      % Using standard european paper size, 'letterpaper' for US letter size
	top=18mm,           % Top margin (it was 18mm before, in case I fucked this up)
	bottom=12.5mm,        % Bottom margin (it was 13mm before, in case I fucked this up) % canonically changed to 12.5
%	left=21mm,          % Left margin
%	right=21mm,         % Right margin
    inner=19mm,         % inner margin
    outer=21mm,         % outer margin
	headheight=19mm,    % Header height
	%includehead,       % -- Without these, the header and footer are --
	%includefoot,       % -- not included within the area defined by margins --
	%footskip=14mm,     % Space from the bottom margin to the baseline of the footer
	%headsep=10mm,      % Space from the top margin to the baseline of the header
	%showframe,         % Uncomment to show how the type block is set on the page
}

\usepackage{xcolor}     % Required for specifying colors by name
\definecolor{mainoffwhite}{RGB}     {244, 239, 230}     % Main colour for pages and some text
\definecolor{corneryellow}{RGB}     {250, 183, 84 }     % Corners, some important things
\definecolor{purplevivid}{RGB}      {162, 92 , 156}     % Inserts
\definecolor{purplight}{RGB}        {180, 149, 184}     % Inserts highlights
\definecolor{somegrey}{RGB}         {106, 109, 114}     % Some stuff's backdrop
\definecolor{almostblack}{RGB}      {38 , 37 , 44 }     % main text colour
\definecolor{prettygreen}{RGB}      {101, 194, 149}     % A pretty green for highlighting text
\definecolor{chalgreen}{RGB}        {142, 211, 124}     % Background color for Challenge page
\definecolor{watermelony}{RGB}      {234, 80 , 151}     % Watermelony Challenge color

\usepackage[pagecolor={mainoffwhite}, nopagecolor={mainoffwhite}]{pagecolor} % provides \thepagecolor, better usage for colour consistency

\usepackage{fancyhdr}   % Fancy headers and footers!
\pagestyle{fancy}       % Set page style to use fancyhdr
\fancyhf{}

\renewcommand\chaptermark[1]{\markboth{\headingfont #1}{}}

\newsavebox{\myheadbox }% Heading storage box for centering headers in yellowcorner


\renewcommand\headrulewidth{0pt} % no line between document and header
\fancyhead{} % clear header
%\fancyhead[RE]{\nouppercase\headingfont\bfseries\fontsize{16pt}{16pt} \leftmark}
\fancyhead[RE]{\savebox{\myheadbox}{\hspace{75mm}}\makebox[\wd\myheadbox][c]{\nouppercase\headingfont\bfseries\fontsize{16pt}{16pt} \leftmark}} %\usebox{\myheadbox}
\fancyfoot{} % clear footer
\fancyfoot[RE,LO]{\headingfont\bfseries\fontsize{10pt}{10pt} Page \thepage}

\usepackage{CharisSIL}

\usepackage{float}                              % manipulate float placement
\usepackage{tikz, tikz-qtree}                   % Required to draw custom shapes & trees
\usepackage{Required/tikzvowel}                 % Vowel charts w/ tikz
\usepackage{pgf}
\usetikzlibrary{positioning, arrows.meta}
\usetikzlibrary{backgrounds, matrix, chains, positioning, decorations.pathreplacing, arrows}
\usetikzlibrary{shapes, arrows, positioning, calc, chains, scopes}
\usepackage{afterpage}                          % For changing page colors (for cover designs)

% needed for Baerian Notation (https://github.com/LLBlumire/recursive-baerian-phonotactics-notation/blob/master/Recursive_Baerian_Syntax_Notation.pdf)
%\usepackage{amsmath}
\usepackage{mathtools}
\usepackage{mathabx}
\usepackage{relsize}
\usepackage[nogroupskip,nonumberlist]{glossaries}
\usepackage{trimspaces}
\usepackage{glossary-mcols}

\usepackage{multicol}

\usepackage{array}                                          % Enables the following,
\newcolumntype{$}{>{\global\let\currentrowstyle\relax}}     % which is really just some automated formatting
\newcolumntype{^}{>{\currentrowstyle}}                      % for tables to have their headers in bold
\newcommand{\rowstyle}[1]{\gdef\currentrowstyle{#1}%
  #1\ignorespaces
}

% Required for including pictures
\usepackage{graphicx}
\graphicspath{ {./Pictures/} }
\usepackage{background}             % Required to draw and plac eiamges in background

\usepackage{ifthen}                 % Conditions!

\usepackage{lingmacros}     % Various useful things some may be used to when writing for conlanging
\usepackage{enumitem}       % Better lists
\setlist{noitemsep,nosep}   % Reduce spacing between bullet points and numbered lists, and between the items

\usepackage{lipsum}         % Inserts dummy text

\usepackage{titlesec}       % Allows to style the titles of sections and all
\usepackage{titletoc}       % Allows manipulation of the ToC

% baarux package and setup from u/akamchinjir
\usepackage[mcolblock,nostandards]{Required/baabbrevs}
\usepackage[canonical,nochapterlabels]{Required/baarux-0.9.11}
\baaruset{junctureoutersep=0.4ex, junctureinnersep=0.2ex}
\baaruset{hangindent=1.5em} % adjusting indenting on multiline glosses, specifically for HidingWaters
\renewcommand{\glossarypreamble}{\thispagestyle{empty}} % for removing page number from abbreviations glossary

\newbaarulinetype{tr}{lit}{\textit{lit.} \baarudquote}
\newbaarulinetype{tr}{alttr}{Or: \baarudquote}

% sets example counter to only count within chapters, but not show the chapter number
%\counterwithout{exesi}{chapter} % Removed this, added [nochapterlabels] to package options instead

\usepackage{verbatimbox}

\usepackage{tabu} % Tables!

\usepackage[hidelinks,hypertexnames=false]{hyperref} % Links will not alter the format

\usepackage{bookmark} % Allows for pdfbookmark

\usepackage[scaled=.92]{helvet}%. Sans serif - Helvetica
\usepackage{calc, color}

\usepackage{multirow} % For annoying tables

\usepackage{ulem}   % for better, customisable underlines


%------------------------------------------------------
%   FRILL & EMBELLISHMENTS
%------------------------------------------------------

\newcommand{\separ}{
	\begin{center}
		\noindent\rule{0.7\textwidth}{0.4pt}
	\end{center}
}

\newfontfamily\playfair[Path=Required/Fonts/]{PlayfairDisplay.ttf}
\DeclareTextFontCommand{\pffont}{\playfair}

\newfontfamily\latolight[Path=Required/Fonts/]{Lato-Light.ttf}
\DeclareTextFontCommand{\llfont}{\latolight}

\newfontfamily\nototc[Path=Required/Fonts/]{NotoSerifTC-Medium.otf}
\DeclareTextFontCommand{\tcfont}{\nototc}

\newfontfamily\karma[Path=Required/Fonts/]{Karma-Medium.ttf}
\DeclareTextFontCommand{\kmfont}{\karma}

\usepackage{xeCJK} % for East Asian scripts, must be loaded with XeLaTeX

\usepackage{tabto}

\NumTabs{5}

\newcommand{\scell}[2][c]{%
  \begin{tabular}[#1]{@{}c@{}}#2\end{tabular}} % Used for line breaks within a cell; [t] aligns with top, [b] aligns with bottom

%------------------------------------------------------
%   TABLE OF CONTENTS & BAABBREVS GLOSSARY
%------------------------------------------------------
\usepackage{pdfpages} % for inserting full-page pdf?

\usepackage{tocloft}

\makeglossaries

\makeatletter
\renewcommand{\contentsname}{Showcases}
\renewcommand{\cftchapteraftersnum}{ | }
\renewcommand{\cftchapterfont}{\headingfont}
\renewcommand{\cftchapterleader}{\bfseries\cftdotfill{\cftdotsep}}
\renewcommand{\cftchapterpagefont}{\headingfont}
\makeatother

\newlength{\mylen}   % a "scratch" length
\settowidth{\mylen}{\cftchapterpresnum\cftchapteraftersnum} % extra space
\addtolength{\cftchapternumwidth}{\mylen} % add the extra space


%------------------------------------------------------
%   TITLE STYLING
%------------------------------------------------------

% \newfontfamily\headingfont[]{Arial} % creates a new font command we'll use for titles and headings
% \newfontfamily\latofont[]{Lato}     % creates a new font command we'll use for author info
\newfontfamily\headingfont[]{NotoSans-Regular} % creates a new font command we'll use for titles and headings
\newfontfamily\latofont[]{NotoSans-Regular}    % creates a new font command we'll use for author info
\newcommand{\titlesize}{\fontsize{32pt}{32pt}\selectfont}   % creates a command to set font size to 32pt
\newcommand{\authorsize}{\fontsize{16pt}{16pt}\selectfont}  % creates a command to set font size to 16pt



\makeatletter
\def\@maketitle{
  %\newpage
  \null
  %\vskip .5em
  \begin{center}
  %\let \footnote \thanks
    {\headingfont\titlesize\bfseries \@title \par}
    \vskip .5em
    {%\large
      %\lineskip .5em
      \begin{tabular}[t]{c}
        {\latofont\authorsize\bfseries by \textit{\@author}}
      \end{tabular}\par}
    \vskip 1em
    %{\large \@date}
    %\vskip 1em
    {\headingfont\authorsize\bfseries \subtitle}
  \end{center}
  \par
  \vskip 1.5em}
\makeatother

%---------------
%   TITLE
%---------------


%---------------
%   SUBTITLE
%---------------



%---------------
%   AUTHOR
%---------------


%------------------------------------------------------
%   GRAPHICS
%------------------------------------------------------



%---------------
%   HEADER
%---------------

\renewcommand{\headrulewidth}{0.0pt}

% Defining shapes with tikz
\tikzset{
    hdrmain/.style={ % requires library shapes.geometric
        draw,
        fill=corneryellow,
%        inner sep=mm,
%        outer sep=0mm,
        trapezium,
        trapezium left angle=145,
        trapezium right angle=90,
        minimum height=13mm,
        minimum width= 123mm,
%        text width=45mm,
        text centered,
    },
}
%        \begin{tikzpicture}
%            \node[hdrmain](hdr){\color{almostblack}Plenty of stuff};
%        \end{tikzpicture}

\newif\ifBgUse % Lys's test for removing bg from pages

\backgroundsetup{scale=1,angle=0,opacity=1,contents={}}
\AddEverypageHook{%
    \ifBgUse % Lys's test for removing bg from pages
    \backgroundsetup{contents={}
}%  Set an if/else conditional statement that checks whether the page is odd or not
    \ifthenelse{\isodd{\value{page}}}
        {% IF YES
            \SetBgPosition{current page.north west}
            \SetBgContents{
                \begin{tikzpicture}[remember picture,overlay,xshift=0mm,yshift=0mm]
                    \fill[color=corneryellow](0mm,-297mm) -- (0mm,-290.5mm) -- (95.8mm,-290.5mm) -- (105mm,-297mm) -- cycle;
                \end{tikzpicture}
            }
            \BgMaterial%
        }%
        {% IF NOT
            \SetBgPosition{current page.north east}
            \SetBgContents{
                \begin{tikzpicture}[remember picture,overlay,xshift=0mm,yshift=0mm]
                    \fill[color=corneryellow](0mm,0mm) -- (0mm,-13mm) -- (-105mm,-13mm) -- (-123.5mm,-0mm) -- cycle;
                    \fill[color=corneryellow](0mm,-297mm) -- (0mm,-290.5mm) -- (-52.5mm,-290.5mm) -- (-61.7mm,-297mm) -- cycle;
                \end{tikzpicture}
            }
            \BgMaterial%
        }%
        \fi
}



%---------------
%   CHALLENGE PAGE BACKGROUND STYLE
%---------------

\newif\ifBgUseTwo % Lys's test for removing bg from pages

\backgroundsetup{scale=1,angle=0,opacity=1,contents={}}
\AddEverypageHook{%
    \ifBgUseTwo % Lys's test for removing bg from pages
    \backgroundsetup{contents={}
}%  Set an if/else conditional statement that checks whether the page is odd or not
    \ifthenelse{\isodd{\value{page}}}
        {% IF YES
            \SetBgPosition{current page.north west}
            \SetBgContents{
                \begin{tikzpicture}[remember picture,overlay,xshift=0mm,yshift=0mm]
                    \fill[color=chalgreen](0mm,-297mm) -- (0mm,-290.5mm) -- (95.8mm,-290.5mm) -- (105mm,-297mm) -- cycle;
                \end{tikzpicture}
            }
            \BgMaterial%
        }%
        {% IF NOT
            \SetBgPosition{current page.north east}
            \SetBgContents{
                \begin{tikzpicture}[remember picture,overlay,xshift=0mm,yshift=0mm]
                    \fill[color=chalgreen](0mm,0mm) -- (0mm,-13mm) -- (-105mm,-13mm) -- (-123.5mm,-0mm) -- cycle;
                    \fill[color=chalgreen](0mm,-297mm) -- (0mm,-290.5mm) -- (-52.5mm,-290.5mm) -- (-61.7mm,-297mm) -- cycle;
                \end{tikzpicture}
            }
            \BgMaterial%
        }%
        \fi
}


%------------------------------------------------------
%   HEADINGS STYLING
%------------------------------------------------------

\newcommand{\sectionnamesize}{\fontsize{40pt}{40pt}\selectfont}
\newcommand{\partsize}{\fontsize{32pt}{32pt}\selectfont} % Creates a command that sets font size to X
\newcommand{\chaptersize}{\fontsize{26pt}{26pt}\selectfont} % Creates a command that sets font size to Y
\newcommand{\headingonesize}{\fontsize{18pt}{18pt}\selectfont} % Creates a command that sets font size to 18
\newcommand{\subheadingsize}{\fontsize{15pt}{15pt}\selectfont} % Creates a command that sets font size to 15
\newcommand{\subsubheadingsize}{\fontsize{13pt}{13pt}\selectfont} % Creates a command that sets font size to 15

\newcommand{\segsize}{\fontsize{64pt}{64pt}\selectfont} % command for Segments title size
\newcommand{\issuesize}{\fontsize{20pt}{20pt}\selectfont}

%---------------
%   CHAPTER
%---------------

\makeatletter
  \renewcommand{\thechapter}{\two@digits{\value{chapter}}}
\makeatother

\makechapterstyle{box}{
  \renewcommand*{\printchaptername}{}
  \renewcommand*{\chapnumfont}{\headingfont\bfseries\fontsize{40pt}{40pt}\selectfont}
  \renewcommand*{\printchapternum}{
    \flushleft
    \begin{tikzpicture}
      \draw[fill,color=almostblack, xshift=-26mm] (0mm,0mm) rectangle (10.5mm, 26mm); % big box
      \draw[color=almostblack, xshift=-12mm] (10.5mm,13mm) node { \chapnumfont\thechapter }; % chapter number
      \draw[fill, color=almostblack, xshift=13mm] (0mm, 0mm) rectangle (0.8mm,26mm);   % separator
      %\draw[color=almostblack, xshift=26mm] (26mm,13mm) node {\printchaptertitle{test}};
    \end{tikzpicture}
  }
  \renewcommand*{\afterchapternum}{}
  \renewcommand\printchaptertitle[1]{\headingfont\chaptersize\bfseries%
    \hskip6.5mm \raisebox{10.5mm}{\parbox[s]{\textwidth-55mm}{##1}}
  }
}

\chapterstyle{box}
\newcommand{\byline}[1]{% command to add a by-line
\hskip15mm \raisebox{20mm}{\parbox{\textwidth-55mm}{\headingfont\subheadingsize by \textbf{#1}}}
\newline
}
\newcommand{\articlesub}[1]{% command to make article subtitles with the purple box (since subtitles don't seem to fit nicely with chapterstyles)
\begin{tikzpicture}
  \draw[fill,color=mainoffwhite] (0mm,0mm) rectangle (1mm,1mm);
  \draw[fill,color=purplevivid] (\textwidth+21mm,0mm) rectangle ++(-10.5mm, 13mm);
  \node[anchor=east] at (\textwidth+0mm, 5.5mm) {\headingfont\bfseries\headingonesize #1}; 
\end{tikzpicture}
}
\newcommand{\chalsub}[1]{% command for box color for challenges)
\begin{tikzpicture}
  \draw[fill,color=mainoffwhite] (0mm,0mm) rectangle (1mm,1mm);
  \draw[fill,color=watermelony] (\textwidth+21mm,0mm) rectangle ++(-10.5mm, 13mm);
  \node[anchor=east] at (\textwidth+0mm, 5.5mm) {\headingfont\bfseries\headingonesize #1}; 
\end{tikzpicture}
}



%---------------
%   SECTION
%---------------

\titleformat{\section} % we want to style the section headings only
  {\headingfont\bfseries\headingonesize}{\thesection}{1em}{}
  \titlespacing{\section}{0em}{0em}{0em}

%---------------
%   SUBSECTION
%---------------

\titleformat{\subsection} % we want to style the subsection headings only
  {\headingfont\bfseries\subheadingsize}{\thesection}{1em}{}
  \titlespacing{\subsection}{6.5mm}{0em}{0em}
 
\titleformat{\subsubsection} % we want to style the subsubsection headings only
  {\headingfont\bfseries\subsubheadingsize}{\thesection}{1em}{}
  \titlespacing{\subsubsection}{6.5mm}{0em}{0em}
  
%------------------------------------------------------
%   TEXT STYLING
%------------------------------------------------------

\setlength{\parindent}{3.2mm}   % sets the indentation of the first line of a paragraph
\setlength{\parskip}{12pt}      % sets the space between the paragraph and what's above it
\color{almostblack}

\newcommand{\highlight}[1]{{\color{purplevivid}{#1}}}

%------------------------------------------------------
%   ARTICLE-SPECIFIC STUFF
%------------------------------------------------------

% reduplication
\newbaarucmd[bits]{^}{\baarujuncture{\~{}}}

% infixes
\newbaarucmd[bits]{<}{\baaruopenbracket〈}
\newbaarucmd[bits]{>}{\baaruclosebracket〉}

% smoyding

\newcommand{\dex}[2]{\textbf{#1} \textit{`#2'}}
\def\baarusmoyd#1{(\textsc{5moyd} \##1)}
\newbaarulinetype{citation}{smoyd}{\baarusmoyd}

% ipa glossing
\def\baarubrack#1{[#1]}
\newbaarulinetype{unaligned}{ipa}{\baarubrack}

%------------------------------------------------------
%   LABELS & REF -- KUDOS TO AKAM
%------------------------------------------------------

% Wrap each article in \begin{article}{PREFIX}...\end{article}. PREFIX will
% be added to each cross-referencing label defined or used within the
% article. Two hooks, \PREFIXdefs and \endPREFIXdefs, are run at the start
% and end of the environment. The first can be used to load article-specific
% definitions or configuration, the second is probably useless.

% \makeatletter
% \AtBeginDocument{%
%   \let\seg@ref@saved\ref
%   \let\seg@label@saved\label
%   \let\seg@hyperref@saved\hyperref

%   \def\ref{\@ifstar\seg@ref@starred\seg@ref@}
%   \def\seg@ref@starred#1{\seg@ref@saved*{\segprefix#1}}
%   \def\seg@ref@#1{\seg@ref@saved{\segprefix#1}}

%   \def\label#1{\seg@label@saved{\segprefix#1}}

%   \def\hyperref{\@ifnextchar[\seg@hyperref@optarg\seg@hyperref@saved}%]
%   \def\seg@hyperref@optarg[#1]{\seg@hyperref@saved[\segprefix#1]}
% }

% \newenvironment{article}[1]{%
%   \ifcsdef{segprefix:#1}{%
%     \PackageError{segments}{Duplicate article prefix `#1'}\@eha
%   }{}%
%   \global\cslet{segprefix:#1}\relax
%   \def\segprefix{#1}%
%   \csuse{#1defs}%
% }{%
%   \csuse{end\segprefix defs}%
% }%
% \makeatother
